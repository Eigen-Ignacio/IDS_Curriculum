\section{Mathematics Program}
[Statement]

\subsection{Program Outcomes}
Program outcomes in mathematics refer to the overarching goals and skills that students are expected to achieve upon completion of a mathematics program. These outcomes provide a comprehensive view of the knowledge, abilities, and competencies that students should have acquired during their mathematical education. Here are the program outcomes for mathematics: 
\begin{enumerate}[label=\Alph*.]
	\item{Mathematical proficiency to prepare for future studies:}
	\begin{itemize}
		\item Apply advanced mathematical concepts to develop innovative solutions.
		\item Build a solid foundation in algebra, geometry, and other core areas to succeed in higher-level mathematics courses.
		\item Develop skills that are valuable for success in college and future careers.
	\end{itemize}
	\item{Creative and critical thinking}
	\begin{itemize}
		\item Demonstrate divergent thinking skills in answering open-ended tasks. 
		\item Evaluate mathematical arguments and proofs for validity and logical coherence.
		\item Synthesize information from various sources to approach challenging mathematical problems.
		\item Develop critical thinking skills by analyzing and solving mathematical problems using different strategies.
	\end{itemize}
	\item{Mathematical Communication}
	\begin{itemize}
		\item Effectively communicate mathematical ideas through written reports, presentations, and discussions.
		\item Justify mathematical solutions through clear and coherent explanations.
		\item Present mathematical findings and arguments to peers and teachers.
	\end{itemize}
\end{enumerate}

\subsection{Outcome Mapping}
The above mentioned Math program outcomes map themselves to the overall IDS Outcomes in the following manner:
\begin{center}
	\begin{tabular}{ l | c | c | c }
		IDS Outcomes				& Outcome A 	& Outcome B   &  Outcome C	\\
		1. Academically Prepared	& \checkmark	& \checkmark  &  \checkmark	\\
		2. Holistically Developed	& \checkmark	&			  &	\\
		3. Patriotic 				& 				&			  &	\\
		4. Collaborative			& \checkmark	&			  &	\checkmark \\
		5. Adaptable				&				& \checkmark  &	\\
	\end{tabular}
\end{center}



\subsection{Subject Descriptions}
Every year level has one Math subject, each of which is briefly described here. For the full syllabus of each subject, please refer to Appendix A: Subject Syllabi.

\begin{subject}
	Math 7: Elementary Algebra
	\hfill
	(5hrs/week)
\end{subject}
This course is designed to provide an introductory overview of the fundamental concepts of algebra. The primary focus is on developing a solid understanding of key concepts related to sets, real numbers, algebraic expressions, polynomials, linear equations, and linear inequalities. The goal is to prepare students for Mathematics 8 by establishing a strong foundation in these core algebraic principles. This program offers students a comprehensive mathematical framework that is essential for awide range of academic fields and practical decision-making.

\begin{subject} Math 8: Intermediate Algebra
	\hfill
	(3 hrs/week)
\end{subject}
This course emphasizes on the understanding of basic concepts of Systems of Linear Equations and Inequalities, Quadratic Equation and Inequalities , Cartesian Coordinate System, Rational Expressions and Applications as preparation for Math 9.

\begin{subject} Math 8.1 Geometry
	\hfill
	(3hrs/week)
\end{subject}
This course deals with the measurement, properties and relationships of points, lines, angles, and planes.It provides students with the opportunity to develop mathematical reasoning. Reasoning mathematically means developing and testing conjectures through Deduction. Algebra skills will also be integrated throughout each chapter and reinforced within the Geometry exercises.

\begin{subject} Math 9: Advanced Algebra
	\hfill
	(4hrs/week)
\end{subject}
This course is offered to Grade 9 students at the Integrated Developmental School, MSU-IIT. The course includes topics solving quadratic and rational equations, radicals, variations, sequences and series, functions and relations, different kinds of functions and their graphs, inverse functions, polynomial functions, rational functions, and their applications to the real world. It also contains topics in complex numbers.

\begin{subject} Math 9.1: Trigonometry
	\hfill
	(3hrs/week)
\end{subject}
This course is offered to all grade 9 students of the Integrated Developmental School. It includes topics in angle unit conversions, trigonometric functions, trigonometric identities, trigonometric equations and solutions of right and oblique triangles.

\begin{subject} Math 10: Pre-Calculus
	\hfill
	(3hrs/week)
\end{subject}
This course covers the advanced concepts in algebra, methods of proof, mathematical induction and analytic geometry, and their applications. 

\begin{subject} Math 10.1: Introduction to Statistics
	\hfill
	(3hrs/week)
\end{subject}
This is a one-unit credit course which covers basic concepts in statistics such as the nature of probability and statistics, frequency distributions and their graphs, techniques for data description, probability and counting rules, hypothesis testing, testing means. The course prepares and helps students in performing statistical computations and interpreting results in researches. The course uses and provides experiences for the students to use statistical software such as Excel-Data Analysis and Excel-Megastat in the computations.

\begin{subject} Stat100C: Statistics and Probability
	\hfill
	(4hrs/week)
\end{subject}
This course introduces students to the fundamental concepts of statistics and probability. Students will learn how to analyze data, make informed decisions, and understand the principles of probability. The course emphasizes practical applications in real-life scenarios.

\begin{subject} Math210S: Calculus I
	\hfill
	(4hrs/week)
\end{subject}
This course covers the concepts in exponential and logarithmic functions, trigonometric functions and inverse trigonometric functions and their derivatives and integrals and their applications to prepare grade 12 students into sciences and engineering courses.

\begin{subject} Math310S: Calculus II
	\hfill
	(4hrs/week)
\end{subject}
This advanced calculus course for Grade 12 Senior High students delves into the principles of calculus,exploring both differential and integral calculus. Students will engage with topics such as limits,derivatives, integrals, and their applications in real-world scenarios. Emphasis will be placed on developing a profound understanding of calculus concepts and honing problem-solving skills essential for further studies in mathematics and related fields.