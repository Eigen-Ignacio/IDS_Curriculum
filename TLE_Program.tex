\section{Technology and Livelihood Education Program}
Technology and Livelihood Education is a course that IDS offers to students in grades 7 and 8 to ensure that the students can explore, utilize, and learn relevant skills of home economics. It provides the students a chance to investigate concepts, get real-world experience, and practice critical thinking skills in a safe environment.  

\subsection{Program Outcomes}
Upon completing the subject, students will be able to: 

\begin{itemize}[label=PO\arabic*.]
	\item [PO1.] To provide students with fundamental knowledge and skills in livelihood programs that
	have an impact on their way of life
	\item [PO2.] To equip students with relevant competencies gauged through an authentic and timely assessment, that shall make them productive citizens. 
	\item [PO3.] To work individually or as a team to produce an output using an updated technology 
	application in the field of livelihood education.
\end{itemize}


\subsection{Outcome Mapping} 
The abovementioned TLE program outcomes map themselves to the overall IDS outcomes in the following manner: 

\begin{center}
	\begin{tabular}{ l | c | c | c }
		IDS Outcomes				& PO1		 	& PO2		  &  PO3	 \\
		1. Academically Prepared	& \checkmark	& \checkmark  &  		     \\
		2. Holistically Developed	& \checkmark 	&			  &	 \checkmark  \\
		3. Patriotic 				& 				& \checkmark  &				 \\
		4. Collaborative			& 				&			  &	\checkmark	 \\
		5. Adaptable				&				& \checkmark  &	\checkmark	 \\
	\end{tabular}
\end{center}

PO1 is mapped to school outcomes A and B as the nature of learning the basic knowledge and skill in livelihood programs that will have an impact on their life. PO2 is mapped to outcomes A,C, and E since it equips the students with the competencies gauged to authentic and timely assessment to be productive citizens. Finally PO3 is mapped to outcomes B, C, and E since it uses updated technology application in the field of livelihood education in producing an output either individually or collaboratively.

\subsection{Subject Descriptions}
For the entire school year, the Grade 7 and 8 will undergo the TLE subject being briefly described by level below. 

\begin{subject}
	TLE 7: Household Services
	\hfill
	(4hrs/week)
\end{subject}
TLE 7 is the first TLE subject that the students will encounter upon entering the grade 7 level. Its primary focus is on introducing the students to the different activities related to household services that would be the foundation of home management. This subject instills a sense of appreciation for home and family living, housekeeping management, food and nutrition, and meal management among the learners.

\begin{subject}
	TLE 8: Food Production 
	\hfill
	(3hrs/week)
\end{subject}
TLE 8 focuses on Food Production with Entrepreneurship. The students will encounter the proper methods and techniques in preparing, cooking, storing, and packaging different marketable food items. They will produce various types of meal courses such as appetizers, salads and salad dressings, sandwiches, pasta and noodles, cereals, main courses, and desserts. The subject aims to develop student competency and entrepreneurial skills that are essential for livelihood and income-generating opportunities not only for the benefits of the individual, family, and home but also for economic benefits in the community.


