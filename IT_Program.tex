\section{Information Technology (IT) Program}
IDS offers \textbf{Information Technology} subjects at every year level to ensure that all its graduates are suitably equipped to navigate, use, and learn all the technology requirements their future college courses will demand of them. Thus, the IT subjects at every year level offer a hands-on experience of many different technologies, ranging from basic productivity tools to 3D modelling and fabrication to Computer Programming and Robotics.

\subsection{Program Outcomes}
Upon graduating, students should be able to:
\begin{enumerate}[label=\Alph*.]
	\item{Demonstrate effective use of commonly-used technologies across subjects.}
	\item{Demonstrate an openness to learning new technologies in their chosen fields.}
\end{enumerate}

\subsection{Outcome Mapping}
The abovementioned IT program outcomes map themselves to the overall IDS Outcomes in the following manner:
\begin{center}
\begin{tabular}{ l | c | c }
	IDS Outcomes				& Outcome A 	& Outcome B 	\\
	1. Academically Prepared	& \checkmark	&				\\
	2. Holistically Developed	& \checkmark	&				\\
	3. Patriotic 				& 				&				\\
	4. Collaborative			& \checkmark	&				\\
	5. Adaptable				&				& \checkmark	\\
\end{tabular}
\end{center}

Outcome A is mapped to IDS Outcomes 1, 2, and 4 as the nature of learning how to use various technologies as well as applying them to learning activities promotes academic preparedness, holistic development, and collaborative learning. Outcome B is mapped to Outcome 5 since learning all these technologies demands that students be willing to learn new concepts and skills while adapting old ones to new contexts and applications.

\subsection{Subject Descriptions}
Every year level has one IT subject, each of which is briefly described here. For the full syllabus of each subject, please refer to Appendix A: Subject Syllabi.

\begin{subject}
IT 7: Productivity Tools
\hfill
(2hrs/week)
\end{subject}
IT7 is the first IT subject students will encounter upon entering the school at the Grade 7 level. Its primary focus is on introducing students to the various online and offline productivity tools that will be used again and again in many other subjects throughout their stay in IDS, such as Google Docs, Google Sheets, various computational engines, and the like. 

\begin{subject} IT 8: Robotics 1
\hfill
(2hrs/week)
\end{subject}
IT8 is the introduction to the various technologies involved in the field of robotics. Students will learn how to render objects in three-dimensions using Computer-Aided Design(CAD) applications, set-up and run a 3D printer, and learn the basics of computer programming in the C/C++ languages.

\begin{subject} IT 9: Multimedia
\hfill
(2hrs/week)
\end{subject}
IT9 is where students are introduced to the production process of various forms of multimedia, which will be used in many of their other non-IT subjects. Students will learn the fundamentals of digital photography, composition, graphic design, audio recording and engineering, and film making.

\begin{subject} IT 10: Robotics 2
\hfill
(2hrs/week)
\end{subject}
IT10 is the continuation of the technologies involved in Robotics, this time including the basics of electrical safety and electronics, circuit design, and the fabrication, assembly, and control of robots. 

\begin{subject} IT 120A: Empowerment Technologies
\hfill
(4hrs/week)
\end{subject}
This is an SHS subject that seeks to equip students with the technological tools their particular strand requires. The exact topics covered is different between the STEM and ABM strands, but includes the use of typesetting applications, statistical software, and a more in-depth study of computer programming. Note that students in the HUMSS strand do not take subject.

\begin{subject} IT 100C: Media and Information Literacy
\hfill
(4hrs/week)
\end{subject}
This is an SHS subject for the equipping of students with the necessary tools and conceptual framework for questioning, critiquing, and investigating the claims of media and the information being coursed through them. It explores such concepts as Meaning Making, Propaganda, and Online Behaviour to help students be more critical consumers and producers of media.
