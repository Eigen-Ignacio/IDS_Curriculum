\section{Nature of the Program}
The MSU-IIT Integrated Developmental School (IDS) has been in existence since 1960, beginning as part of the Lanao Technical School prior to being absorbed by the institute and renamed as the Developmental High School (DHS). It was renamed IDS in the early 90s and has remained so to this day.

Today, IDS serves secondary-school students in Junior High School (JHS) and Senior High School (SHS), with JHS ranging from Grades 7-10 and SHS encompassing Grades 11 and 12.

This curriculum is proposed in the light of the following:
\begin{itemize}
	\item{The need for IDS to function as the institute's Center for Pedagogical Research, giving ample room and support to educational researchers conducting their studies on instructional methods, materials, school organization, and the like.}
	\item{The expectation from the school's stakeholders to uphold and maintain high academic standards, especially in the areas of Science and Mathematics.}
	\item{The desire to produce secondary-school graduates that possess the desired knowledge, skills, and attitudes as expressed in the institute's desired outcomes.}
\end{itemize}

\subsection{School Outcomes}
The desired outcomes of the school are derived from the desired objectives of the institute, framed in the context of secondary education. Thus, upon completing the IDS program from grades 7 to 12, graduates are expected to be:
\begin{enumerate}
	\item{\textbf{Academically Prepared:} Able to qualify for any university and any degree program they desire and
possessing a solid, integrated foundation in all subject areas in terms of content, skills, and application that can be reliably built-upon and enhanced by their future studies.}
	\item{\textbf{Holistically Developed:} Physically, emotionally, and morally healthier and more mature than when they first entered the school, and able to conduct themselves appropriately, productively, and with integrity in any social context they enter into.}
	\item{\textbf{Patriotic:} Able to articulate both verbally and in writing how they identify with their community and the nation as a whole in terms of history, culture, and shared values, all in a guided democratic manner and have participated in at least one service program organized by the school for the benefit of the community.}
	\item{\textbf{Collaborative:} Able and willing to work harmoniously with others who are different from themselves toward meaningful and productive ends.}
	\item{\textbf{Adaptable:} Able to make the necessary cognitive, behavioral, and emotional adjustments in order to function well in the face of new, uncertain, and constantly changing situations both inside and out of the classroom.}
\end{enumerate}

\subsection{College-Entry Options}
The IDS Program aims to prepare all its graduates for \textbf{any undergraduate-level program in any university they desire} by equipping and ensuring they possess a solid grasp of the fundamental principles of mathematics, natural sciences, language, the arts and humanities, and sports so that they are able to learn whatever advanced skills, knowledge, and attitudes their chosen college degree programs demand of them.