\section{Rationale, Objectives, Strategies, and Processes}
\begin{tabular}{l c l}
	\textbf{College} 		&	:  &	College of Education\\
	\textbf{Department}	&	:  &	Integrated Developmental School\\
	\textbf{Branch/Unit}&	:  &	Mindanao State University-Iligan Institute of Technology\\
	\textbf{Proposal}		&	:  &	Revision of the G7-12 Curriculum of the Integrated Developmental School\\
\end{tabular}

\subsection{Rationale}
With the many changes brought about by the K to 12 program of the Department of Education (DepEd) as well as subsequent developments to the structure of the College of Education (CED) and the institute at large, it is high-time that the Integrated Developmental School (IDS) proactively revise its curriculum, aligning it with clear and achievable goals and outcomes.

Despite the many decades of its existence, there has been no official "IDS curriculum" that the institute can point to, evaluate, and assess. The vast majority of policies in the school are the result of ad-hoc decisions made years, if not decades ago that have simply been passed-down as a kind of tradition. While this is not necessarily negative, a clear, intentional, and most of all \textit{official} corpus of curriculum policies is now in order, as embodied by this proposal.

Specifically, this revision aims to:
\begin{enumerate}
	\item{Concretize the school's function as the institute's Center for Pedagogical Research, giving ample room and support to educational researchers conducting their studies on instructional methods, materials, school organization, and the like.}
	\item{Meet and even exceed expectations from the school's stakeholders to uphold and maintain high academic standards, especially in the areas of Science and Mathematics.}
	\item{Clarify the curriculum offerings of the IDS secondary school program from grades 7 to 12 in light of the school's role as the center for pedagogical research of the institute.}
	\item{Provide clarity to CED and the institute at large regarding the nature and details of the IDS secondary school program.}
	\item{Provide a clear and objective basis for assessment, evaluation, and future revisions and improvements.}
\end{enumerate}



\subsection{Program Features}
The proposed curriculum of the IDS secondary school program has the following salient features:
\begin{enumerate}[label=(\Alph*)]
	\item{A complete map of all subjects from Grades 7 to 10 in Junior High School (JHS) and Grades 11 and 12 in Senior High School (SHS), across all offered SHS strands.}
	\item{Content offering across all grade levels that exceed the minimum standards set by DepEd.}
	\item{Complete acceptance, retention, and honors policy statements for JHS and SHS.}
\end{enumerate}



\subsection{Objectives of the Program}



\subsection{Program Outcomes}
The desired outcomes of the school are derived from the desired objectives of the institute, framed in the context of secondary education. Thus, upon completing the IDS program from grades 7 to 12, graduates are expected to be:
\begin{enumerate}
	\item{\textbf{Academically Prepared:} Able to qualify for any university and any degree program they desire and
possessing a solid, integrated foundation in all subject areas in terms of content, skills, and application that can be reliably built-upon and enhanced by their future studies.}
	\item{\textbf{Holistically Developed:} Physically, emotionally, and morally healthier and more mature than when they first entered the school, and able to conduct themselves appropriately, productively, and with integrity in any social context they enter into.}
	\item{\textbf{Patriotic:} Able to articulate both verbally and in writing how they identify with their community and the nation as a whole in terms of history, culture, and shared values, all in a guided democratic manner and have participated in at least one service program organized by the school for the benefit of the community.}
	\item{\textbf{Collaborative:} Able and willing to work harmoniously with others who are different from themselves toward meaningful and productive ends.}
	\item{\textbf{Adaptable:} Able to make the necessary cognitive, behavioral, and emotional adjustments in order to function well in the face of new, uncertain, and constantly changing situations both inside and out of the classroom.}
\end{enumerate}


\subsection{Implementation Scheme of the Curriculum}




