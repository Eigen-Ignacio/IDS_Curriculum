\section{Rationale, Objectives, Strategies, and Processes}
\begin{tabular}{l c l}
	\textbf{College} 		&	:  &	College of Education\\
	\textbf{Department}	&	:  &	Integrated Developmental School\\
	\textbf{Branch/Unit}&	:  &	Mindanao State University-Iligan Institute of Technology\\
	\textbf{Proposal}		&	:  &	Revision of the G7-12 Curriculum of the Integrated Developmental School\\
\end{tabular}

\subsection{Rationale}
With the many changes brought about by the K to 12 program of the Department of Education (DepEd) as well as subsequent developments to the structure of the College of Education (CED) and the institute at large, it is high-time that the Integrated Developmental School (IDS) proactively revise its curriculum, aligning it with clear and achievable goals and outcomes.

Despite the many decades of its existence, there has been no official "IDS curriculum" that the institute can point to, evaluate, and assess. The vast majority of policies in the school are the result of ad-hoc decisions made years, if not decades ago that have simply been passed-down as a kind of tradition. While this is not necessarily negative, a clear, intentional, and most of all \textit{official} corpus of curriculum policies is now in order, as embodied by this proposal.

Specifically, this revision aims to:
\begin{enumerate}
	\item{Concretize the school's function as the institute's Center for Pedagogical Research, giving ample room and support to educational researchers conducting their studies on instructional methods, materials, school organization, and the like.}
	\item{Meet and even exceed expectations from the school's stakeholders to uphold and maintain high academic standards, especially in the areas of Science and Mathematics.}
	\item{Clarify the curriculum offerings of the IDS secondary school program from grades 7 to 12 in light of the school's role as the center for pedagogical research of the institute.}
	\item{Provide clarity to CED and the institute at large regarding the nature and details of the IDS secondary school program.}
	\item{Provide a clear and objective basis for assessment, evaluation, and future revisions and improvements.}
\end{enumerate}



\subsection{Curriculum Features}
The proposed curriculum of the MSU-IIT Integrated Developmental School has the following salient features:
\begin{enumerate}
	\item{A complete map of all subjects from Grades 7 to 10 in Junior High School (JHS) and Grades 11 and 12 in Senior High School (SHS), across all offered SHS strands.}
	\item{Competencies across all grade levels that exceed the minimum standards set by DepEd.}
	\item{Complete acceptance, retention, and honors policy statements for JHS and SHS.}
\end{enumerate}



\subsection{Outcomes}
The desired outcomes of the school are derived from the desired objectives of the institute, framed in the context of secondary education. Before enumerating and describing them, however, it is important to understand how the terms are defined in the context of IDS in order to understand and facilitate outcome alignment throughout the curriculum.

\textbf{OUTCOMES} refers to the knowledge, skills, and attitudes that the students are expected to possess and exhibit \textit{after} going through the subject, program, or the IDS curriculum as a whole. They are not statements on what the school will do, but we wish to achieve in the students.

\textbf{SCHOOL OUTCOMES} refers to student knowledge, skills, and attitudes that students of IDS should possess \textit{upon graduation}.

\textbf{PROGRAM} refers to the string of subjects administered by a particular department. For example, IT7, IT8, IT9, IT10, IT120A, and IT100C form the IT Program of the School, which is administered by the Math, IT, and ABM (MITABM) department. The rationale behind the use of this terminology is because each subject is the prerequisite of the next, making each subject part of an overall educational program in that content area.

\textbf{PROGRAM OUTCOMES} refers to the knowledge, skills, and attitudes that students should possess after they pass and complete \textit{all} the subjects in a particular program. This is not always "upon graduation" as some programs in the school (such as Technology and Livelihood Education) terminate in JHS.

\textbf{SUBJECT} refers to a particular subject taken during a particular year level. For example, Math 7, which is the math subject taken by all students in the 7th Grade.

\textbf{SUBJECT OUTCOMES} refers to the knowledge, skills, and attitudes that students should possess after they complete a given subject. This is usually after a particular year level (in the case of JHS), but can also be after a particular semester (such as in SHS).

\subsubsection{School Outcomes}
Upon completing the IDS curriculum, graduates are expected to be:
\begin{enumerate}[label=\Alph*. ]
	\item{\textbf{Academically Prepared:} Able to qualify for any university and any degree program they desire and possessing a solid, integrated foundation in all subject areas in terms of content, skills, and application that can be reliably built-upon and enhanced by their future studies.}
	\item{\textbf{Holistically Developed:} Physically, emotionally, and morally healthier and more mature than when they first entered the school, and able to conduct themselves appropriately, productively, and with integrity in any social context they enter into.}
	\item{\textbf{Patriotic:} Able to articulate both verbally and in writing how they identify with their community and the nation as a whole in terms of history, culture, and shared values, all in a guided democratic manner and have participated in at least one service program organized by the school for the benefit of the community.}
	\item{\textbf{Collaborative:} Able and willing to work harmoniously with others who are different from themselves toward meaningful and productive ends.}
	\item{\textbf{Adaptable:} Able to make the necessary cognitive, behavioral, and emotional adjustments in order to function well in the face of new, uncertain, and constantly changing situations both inside and out of the classroom.}
\end{enumerate}

The achievement indicators for each of these desired school outcomes, as well as the activities by which these indicators will be assessed are detailed in Table \ref{table:1}.

\subsubsection{Program Outcomes}
There are 12 program areas in IDS, namely: Math, Physics, Chemistry, Biology, IT, English, Filipino, Social Science, Research, Technology and Livelihood Education (TLE), Values, etc.


\vspace{1em}
\begin{table}
\small
\renewcommand{\arraystretch}{1.6}
\begin{tabularx}{\textwidth}{X p{5.7cm} p{6cm}}

\multicolumn{1}{c}{\textbf{School Outcomes}} &	\multicolumn{1}{c}{\textbf{Performance Indicators}} & \multicolumn{1}{c}{\textbf{Assessment}}\\
\toprule
A. Academically Prepared & Acceptance into at least one desired university. & Letter of Acceptance from at least one university; OR\\

& & Student's name appears on the official list of passers of at least one desired university.\\
\midrule
B. Holistically Developed	& More physically-fit upon graduation compared to first enrollment. & Improving test scores in annual fitness examinations.\\

& & Participation in at least one extracurricular club between Grades 7--12.\\
\midrule
C. Patriotic	&	Able to articulate verbally and in writing how being Filipino is relevant to their sense of identity and belonging & Pertinent writings and recitations in Filipino, English, and Social Studies subjects.\\

& & Participation in at least one community service program organized by the school.\\
\midrule
D. Collaborative & Able and willing to work productively with others, even if they differ in ability, background, religious beliefs, etc. & Subject-embedded groupwork and projects with peer evaluations.\\
\midrule
E. Adaptable & Can make the needed cognitive, behavioral, and emotional adjustments both in and out of the classroom & Passing marks in all subjects from Grade 7--12.\\
\bottomrule
\end{tabularx} 
\caption{\textit{School Outcomes, Indicators, and Assessments}}
\label{table:1}
\end{table}

\subsection{Implementation Scheme of the Curriculum}
The following section defines and elaborates on the school policies regarding the admission, retention, and graduation of students from the curriculum. 

\subsubsection{Admission Requirements}
As of Academic Year (AY) 2023-2024, following the discontinuation of the SHS voucher program of DepEd and the renewed emphasis by the Commission on Higher Education (CHED) on the population limits for laboratory schools operated by State Universities and Colleges (SUCs), the only point of admission into the school is at Grade 7, which has the following requirements:
\begin{itemize}
	\item{Inclusion in the official list of passers of the IDS Entrance Examination for the upcoming academic year.}
	\item{Two (2) photocopies of the student's Grade 6 report card from their previous school.}
\end{itemize}


