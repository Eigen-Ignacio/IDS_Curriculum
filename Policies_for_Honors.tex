\section{Policies for Honors}
To encourage students to aim for excellence, recognition is given to those who performed well in academic and non-academic endeavors in and out of the campus. These students are recognized at the end of each of the first three quarters and at the end of the school year.

\begin{enumerate}[label=(\Alph*)]
	\item \textbf{JUNIOR HIGH SCHOOL}
	
	\begin{enumerate}[label=\Alph{enumi}.\arabic*]
		\item \textbf{Quarterly Awards} \\
		During the Quarter Recognition Awards, the following Awards are given:
		
		\textbf{Academic Honors} are awarded to students who excelled in their studies for the given quarter. To qualify for the Honors Roll, a student must have no grade below 80\% in any subject (including the MAPEH components) and no grade below 85\% in any character grade for the given quarter. Furthermore,he must have not committed any Category C offenses for the quarter.
		\begin{itemize}
			\item The \textbf{First Honors Award} is given to students whose grade point average for the quarter is 94\% or better.
			\item The \textbf{Second Honors Award} is given to students whose grade point average for the quarter is from 92\% to 93.99\%.
			\item The \textbf{Third Honors Award} is given to students whose grade point
			average for the quarter is from 90\% to 91.99\%.
		\end{itemize}
		
		\textbf{Co-Curricular} and \textbf{ Extra-Curricular Awards} are given to students who won in off-campus competitions officially recognized by the school during the quarter.
		
		\textbf{These Awards are given during the 1st, 2nd, and 3rd quarters.}
		
		\item \textbf{Academic Year-End Awards}
		
		At the end of the school year, the same Academic Honors are awarded to the students during the year-end recognition activities. To qualify for the year-end honors roll, a student must have no final grade below 80\% in any subject and in any character grade. Moreover, the student must have no grade below 80\% in any subject and no grade below 85\% in any character grade during the previous quarters. Also, he must have not committed any Category C offenses for the school year.
		
		\textbf{Academic Excellence Awards} \\
		The Academic Excellence Award is given to students who earned the highest academic grade in a particular subject. For the Science and Math where two or more subjects are offered in a year level, the weighted average of the final grades will be used as the basis for the selection of the awardee. 
		
		\textbf{Service Awards} \\
		The Service Award is given to students from any grade level who have rendered exemplary service to the school for the current school year. This particular service must be outside of the expectations of any office held. This award is given upon the recommendation of any faculty member, and need not be awarded every year.
		
		\item \textbf{Completion Awards}
		
		\begin{enumerate}[label=\arabic*.]
			\item The following honors are given during the completion ceremonies
			\begin{center}
				\begin{tabular}{@{}l@{}}
					\textbf{With Highest Honors} \\
					\textbf{High High Honors}\\
					\textbf{With Honors}\\
					\textbf{With Distinction}
				\end{tabular}
			\end{center}
			
			The aforementioned awards will be given to any member of the graduating class who
			\begin{enumerate}[noitemsep]
				\item[a.] does not have any failing grade in any subject in any grading period
				(Grade 7-10) 
				\item[b.] does not have any grade below 80 in their 9th and 10th grades. 
				\item[c.] has completed the last two years of junior high school in IDS 
				\item[d.] has completed the junior high school academic program for four
				consecutive years. 
				\item[e.] has not committed any of the category C offenses for the school year.
				\item[f.] will be ranked according to their overall weighted average (OWA). \\
				
				The OWA will be computed using the formula shown below:
				
				\begin{equation}
					OWA = \frac{\text{\textit{WA Grade 9}} + \text{\textit{WA of 3 grading periods in Grade 10}}}{2}
				\end{equation}
			\end{enumerate}
			
			Where: 
			\begin{align*}
				OWA &= \text{\textit{Overall Weighted Average}} \\
				WA &= \text{\textit{Weighted Average}}
			\end{align*}
			
			\item The following criteria and the corresponding relative weights shall be used in determining the With Highest Honors, With High Honors, With Honors and With Distinction awards.
			
			\vspace{-0.5cm}
			
			\textbf{
			\begin{enumerate}[noitemsep]
					\item[a.] \begin{tabular}[t]{@{}l@{\hspace{11.5em}}l@{}}
						Academic Excellence & $70\%$ \\
					\end{tabular}
					\item[b.] \begin{tabular}[t]{@{}l@{\hspace{2em}}l@{}}
						Co-Curricular/Extra Curricular Excellence & $30\%$ \\
					\end{tabular}
			\end{enumerate}}
			
			The computed values for the OWA are considered as the candidates’Academic Excellence points while the Co-Curricular and Extra Curricular Excellence points are based on their involvement in non-academic pursuits. Hence, completion candidates are expected to submit supporting documents for their community involvement and active participation in authorized student organizations or clubs, literary-musical and athletic activities in the class, year levels, school as well as participation in citywide, regional, inter-regional, national and international contests as approved by the school.
			
		\end{enumerate}
	\end{enumerate}
	
	\item  \textbf{SENIOR HIGH SCHOOL}
	
	
\end{enumerate}